\hypertarget{index_Overview}{}\section{Overview}\label{index_Overview}
Path planning for a point robot using Rapidly Exploring Random Trees (\hyperlink{classRRT}{R\+RT}) on a known 2D space. The algorithm returns coordinate points in the path, which when interfaced with a simple position control system can be used to drive a robot in the planned path. Path R\+R\+Ts are kinodynamic planners that can be used to calculate the trajectory of a robot in real time Given that the algorithm uses incremental motions, it can be used in Collision detection. The \hyperlink{classRRT}{R\+RT} algorithm can be used to produce good guesses for variational optimization techniques.\hypertarget{index_RRT}{}\section{A\+L\+G\+O\+R\+I\+T\+HM}\label{index_RRT}
1.\+Sample a random point from the configuration space. 2.\+Obtain a point on the tree closest to the sampled point, in the direction of the point at a unit distance. 3.\+Verify if this point is in contact with any obstacle. 4.\+If the new point is not in contact with or within any obstacles then add this point to the tree. 5.\+If the new point is in contact with or within any obstacles, then add the point closest to the tree that is just outside the obstacle to the tree. 6.\+The points added to the tree are removed from the sampling space. 7.\+This is recursively performed till the point within unit distance of the goal point is reached.\hypertarget{index_Project}{}\section{Specifics}\label{index_Project}
Programming Language -\/ C++ Build Platform – Make, G\+CC Compiler Source code control -\/ G\+IT and Git\+Hub Build testing – Travis CI Test coverage -\/ Coveralls\hypertarget{index_Agile}{}\section{Process}\label{index_Agile}
\href{https://docs.google.com/spreadsheets/d/1cJVLNv9pZ2T4a17OsMPn_WnxRS6tAkfYJKaMcSRo6MA/edit?usp=sharing}{\tt Sheet Link}\hypertarget{index_Class}{}\section{Diagram}\label{index_Class}
 